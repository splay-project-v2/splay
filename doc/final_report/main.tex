% EPL master thesis covers template
\documentclass{eplmastersthesis}

% Please fill in the following boxes
% Title of the thesis
\title{Integrated mini-cloud of RaspberryPIs for distributed systems training}

% Subtitle - remove this line if not applicable
\subtitle{Subtitle (optional)}

% Name of the student author(s)
\author{Rémy \textsc{Voet}}
\secondauthor{Samuel \textsc{Monroe}}		% remove if not applicable
%\thirdauthor{Firstname \textsc{Lastname}}			% remove if not applicable

% Official title of the master degree (copy/paste from list below)
% Master [120] in Biomedical Engineering
% Master [120] in Chemical and Materials Engineering
% Master [120] in Civil Engineering
% Master [120] in Computer Science
% Master [120] in Computer Science and Engineering
% Master [120] in Cybersecurity
% Master [120] in Data Sciences Engineering
% Master [120] in Data Science: Information technology
% Master [120] in Electrical Engineering
% Master [120] in Electro-mechanical Engineering
% Master [120] in Mathematical Engineering
% Master [120] in Mechanical Engineering
% Master [120] in Physical Engineering
% Master [60] in Computer Science
% Specialised master in nanotechnologies
% Specialised master in nuclear engineering
\degreetitle{Master [120] in Computer Science}

% Name of the supervisor(s)
\supervisor{Firstname \textsc{Lastname}}
\secondsupervisor{Firstname \textsc{Lastname}}		% remove if not applicable
%\thirdsupervisor{Firstname \textsc{Lastname}}		% remove if not applicable

% Name of the reader(s)
\readerone{Firstname \textsc{Lastname}}
\readertwo{Firstname \textsc{Lastname}}			% remove if not applicable
\readerthree{Firstname \textsc{Lastname}}			% remove if not applicable
%\readerfour{Firstname \textsc{Lastname}}			% remove if not applicable
%\readerfive{Firstname \textsc{Lastname}}			% remove if not applicable

% Academic year (update if necessary)
\years{2018--2019}

% Document
\begin{document}
  % Front cover page
  \maketitle

  \chapter*{Abstract}

  \chapter*{Acknowledgements}

  \tableofcontents

  \chapter{Introduction}

    \section{Motivations}

      The process of learning distributed systems and algorithms is usually
      undermined by the difficulty of being able for one to test and apply
      what he learns.\\
      Indeed, in order to run a distributed algorithm and observe the result, a
      collection of machines running the algorithm is needed. Besides that,
      one would like to be able to change the conditions in which the algorithm
      is run, for example by provoking faulty nodes, or inducing network
      perturbations among the system, as distributed algorithms are designed to
      adapt to these conditions. These needs make it really cumbersome
      for students to setup a testing environment emulating realistic
      conditions for learning purposes.\\

      The motivation of the Splay Project is to solve this issue by providing
      professors and students with a whole platform




  % Back cover page
  \backcoverpage

\end{document}
